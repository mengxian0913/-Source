% !TeX encoding = UTF-8
% !TeX program = xelatex
\documentclass[12pt, a4paper]{article}
\usepackage{xeCJK} % 须放在\usepackage{}列中足够前的位置
\usepackage{fontspec}
\usepackage{graphicx}
\usepackage{caption}
\usepackage{enumerate}
\usepackage{setspace}
\usepackage{array} % 製作表格必須的宏包
\usepackage{tabularx} % 自動調整列寬的表格宏包
\usepackage{adjustbox}
\usepackage{listings}
\usepackage{xcolor}
\setCJKfamilyfont{heiti}{Heiti TC}
\CJKfamily{heiti}
\setmainfont{Arial}
\setstretch{1.5}

% 程式碼區塊的樣式設置
\lstset{
    language=C,
    basicstyle=\ttfamily\small,
    numbers=left,
    numberstyle=\tiny,
    frame=single,
    breaklines=true,
    keywordstyle=\color{blue},
    commentstyle=\color{green!60!black},
    stringstyle=\color{red},
    showstringspaces=false,
    tabsize=4,
    inputencoding=utf8  % 添加這行
}


\begin{document}
\begin{center}
  {\Huge 微處理機實習} \\[2.5cm]
  {\Huge Lab8} \\[1.5cm]
  \hspace{.6in}
  \begin{minipage}[t]{.4\linewidth}
    {\Large 班級:資訊三甲}\\[0.5cm]
    {\Large 學號:D1109023}\\[0.5cm]
    {\Large 姓名:楊孟憲}
  \end{minipage}    
\end{center}

\newpage

\begin{description}
  \fontsize{18pt}{22pt}\selectfont 
    \item [一、]實驗目的 \\
      \begin{samepage}
        \fontsize{14pt}{16pt}\selectfont
        此次實驗利用程式邏輯的編寫 操作 GPIO 上的 LCD 畫面捲動和 Keyboard + GPIO 中斷操作。
        \end{samepage}

    \item [二、]遭遇的問題 \\
      \begin{samepage}
        \fontsize{14pt}{16pt} \selectfont
        沒有未能解決的問題。
      \end{samepage}

    \item [三、]解決方法
      \begin{description}
        \fontsize{16pt}{18pt}\selectfont
        \item [$\bullet$]實驗一\hspace{5pt} 畫面捲動 \\
        \lstinputlisting{./sourceCode/p1.c}
        \item [$\bullet$]實驗二\hpace{5pt} 數字會跑步 \\
        \lstinputlisting{./sourceCode/p2.c}
        % 這裡想要插入代碼
      \end{description}
    \item [四、]未能解決的問題 \\[.6cm]
      \begin{minipage}[t]{\linewidth}
        \fontsize{14pt}{16pt}\selectfont
        沒有未能解決的問題。
      \end{minipage}
  \normalsize
\end{description}
\end{document}


